\setcounter{page}{1}
\pagestyle{fancy}
\fancyhf{}
\fancyhead[R]{\thepage}
\renewcommand{\headrulewidth}{0pt} %obere Trennlinie

\section{Introduction}
Nowadays, many processes that handle large amounts of data (not to confuse with Big Data
\footnote{Data with high variety, volume and velocity. Cannot be processed by conventional data processing software.})
already exist. Still, they must be maintained, verified and optimized to ensure both
quality and efficiency.

\subsection{Motivation}
At q.beyond, we had to deal with just that: large monthly CSV exports by a third-party service provider
that our process was operating on. For verification, we collected the exports from last year,
grouped the records by a column and exported each group as an individual Excel file.
Each file contained two sheets: one with an aggregated monthly overview and the other with all the
individual records of that particular group. As we will explore in \ref{Grouping}, this
result actually requires two different kinds of grouping.

[TODO: Result Figure]

We ended up writing a PowerShell\footnote{A scripting language coming with Windows. Used for administrative and automation tasks.}
Script, which, in our first draft, crashed the \gls{AVD}
it was running on due to insufficient \gls{RAM}.

\subsection{Main contributions}
Based on this experience, we can define the purpose of this paper:
\begin{itemize}
    \item \textbf{Problem}: We need to group CSV data bigger than the available RAM
    \item \textbf{Objectives}:
    \begin{enumerate}
        \item Find a solution that can group large CSV data
        \item Proof by tests and simulation that the solution works
    \end{enumerate}
    \item \textbf{Questions}:
    \begin{enumerate}
        \item Why does the \verb+Group-Object+ command not work on limited RAM?
        \item What are requirements for a grouping algorithm in order to run on low RAM?
    \end{enumerate}
\end{itemize}
As an alternative solution, we will consider MySQL, a \gls{DBMS}.

\newpage
\section{Background}

\subsection{Relational Model}

% We will use relational algebra to describe the actual grouping
% operations we want to perform.
% In chapter 4.2.1, \cite{Ram98} presents the selection
% operator, which selects a subset of rows from a relation.

% \begin{definition}
% Let $R$ be a relation. Then
% \[
%     \sigma_{c}({R}) := \set{e \in R | c(e)}
% \]
% is the selection on R with elements for which the condition $c$ holds true.
% \end{definition}

\newpage

\section{Simulation}

\subsection{Methods}
\subsection{Implementation}
\subsection{Results}

\section{Conclusion}
\subsection{Comparison}
\subsection{Future Work}