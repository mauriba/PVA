\setcounter{page}{1}
\pagestyle{fancy}
\fancyhf{}
\fancyhead[R]{\thepage}
\renewcommand{\headrulewidth}{0pt} %obere Trennlinie

\section{Introduction}
Nowadays, many processes that handle large amounts of data (not to confuse with Big Data\footnote{Data
with high variety, volume and velocity. Cannot be processed by conventional data processing software.})
already exist. Still, they must be maintained, verified and optimized to ensure both
quality and efficiency.

\subsection{Motivation} \label{Motivation}
At q.beyond, we had to deal with just that: large monthly CSV exports
(each file contains the data of a month) by a third-party service provider
that our process was operating on. For verification, we collected the exports from a whole year,
grouped the records by a column and exported each group as an individual Excel file.
Each file contained two sheets: one with an aggregated monthly overview and the other with all the
individual records of that particular group.
% For this paper, we will focus on the aggregation.
In \ref{Grouping}, we will explain why those
two sheets require grouping, but only the latter requires every particular record to be stored
(in memory or storage).

% TODO: Result figure

We ended up writing a PowerShell\footnote{A scripting language coming with Windows. Used for administrative and automation tasks.}
Script, which, in our first draft, crashed the \gls{AVD}
it was running on due to insufficient memory.

\subsection{Main contributions}
Based on this experience, we can define the purpose of this paper:
\begin{itemize}
    \item \textbf{Problem}: We need to group CSV data bigger than the available memory
    \item \textbf{Objectives}:
    \begin{enumerate}
        \item Find a solution that can group large CSV data
        \item Proof by tests and simulation that the solution works
    \end{enumerate}
    \item \textbf{Questions}:
    \begin{enumerate}
        \item Why does the \verb+Group-Object+ command not work on limited memory?
        \item What are requirements for a grouping algorithm in order to run on low memory?
    \end{enumerate}
\end{itemize}
As an alternative solution, we will consider MySQL, a \gls{DBMS}.

\newpage
\section{Background}

\subsection{Comma-Separated Values (CSV)}

Specified in \cite{rfc4180}, CSV is a file format for storing a table in plain text.
It has the following characteristics:

\begin{enumerate}
    \item each row starts on a new line
    \item the values for each column are separated by a delimiter (usually a comma)
    \item there may be a header line those values specify the names of the columns
    \item whitespaces other than line breaks are ignored, unless part of a value
\end{enumerate}
Our CSV export is structured in exactly this way. Let it be\footnote{
In this paper, we will use a simplification and abstraction of the original
data mentioned in \ref{Motivation}. The "$\dots$" indicate that
it is indeed a large CSV with many more records.}:

\begin{lstlisting}[escapeinside={(*}{*)},numbers=left,caption=CSV export,label={lst:csvexport}]
tid,    pname,  amount,     comment (*\label{csvexport:header}*)
1,      Alice,  -1,         daily expenses
2,      Bob,    2,          pocket money
3,      Alice,  1,          friend
4,      Carl,   0,          dummy
5,      Dan,    2.99,       card game
...
\end{lstlisting}

%TODO: Add csv export

\subsection{Relational Model}

Introduced by E. F. \cite{Codd70}, the relational model is a mathematical
description of data management, structuring data in tuples and tuples
in relations. A relational algebra is provided to act as a simple
mathematical query language. Industrial query languages
like SQL later implemented the features described in the relational model.

As further explained by \cite{Aren22}, a relation
is just a pair of a relation schema and a set of tuples (the rows of the relation).

\subsubsection{Relational Schema}
Arenas defines the relation schema based on three characteristics:
\begin{itemize}
    \item \textbf{relation name}: the name of the table
    \item \textbf{relation attributes}: a list of uniquely identifiable attributes/columns
    \item \textbf{relation arity}: the number of attributes/columns
\end{itemize}
\cite{Schw10} defines the relation schema more formally as a database-wide mapping
from relation names to either their respective arity (unnamed perspective) or
their respective attributes (named perspective). Then, a row in a relation
is either defined as a tuple of values $a = (a_1,\dots, a_n)$ (unnamed perspective) or
a set of pairs $a = \set{(c_1, a_1),\dots, (c_n, a_n)}$, where each pair consists of
the column name $c_n$ and the value $a_n$ assigned to it (named perspective).

We will use the named perspective. For that, we must ensure that each column
got a unique name. Luckily, the header from our CSV export statisfies this condition
(see listing \ref{lst:csvexport}). In the named perspective, let its schema be:

\begin{center}
\verb+Transaction [ tid, pname, amount, comment ]+
\end{center}

Note that we do not assign any types to the attributes,
as the CSV header row only consists of attribute names.
This conforms with the understanding of the relational model
from Schweikardt, who just assumes that the values for our columns come from an infinite set
containing any values for any columns.

The visual representation of a relation is a table. Below figure shows the
corresponding relation to the first 5 records of the CSV export from listing \ref{lst:csvexport}.

\begin{table}[htbp] \label{Transaction}
    \centering
    \begin{tabular}{|c|c|c|c|}
        \hline
        tid     & pname & amount    & comment \\ \hline\hline
        1       & Alice & -1        & daily expense \\ \hline
        2       & Bob   & 2         & pocket money \\ \hline
        3       & Alice & 1         & friend\\ \hline
        4       & Carl  & 0         & dummy\\ \hline
        5       & Dan   & 2.99      & card game\\ \hline
    \end{tabular}
    \caption{Transaction}
\end{table}

\subsubsection{Selection and Projection}

In chapter 12.1, \cite{Halp08} introduces relational algebra. It includes six
comparison operators $\theta \in \Set{=, <>, <, >, \leq, \geq}$ as well as the
three logical operators $\Set{\land, \lor, \neg}$. Furthermore, some
language-specific operations are defined. We will focus on the selection and projection.

Given a relation $R$, the selection operator $\sigma_c(R)$ selects all rows from
that relation that match the condition $c$ (similar to the \textbf{where} clause in \gls{SQL}).
Such condition can consist of any valid
expression using comparison operators, logical operators, attributes and constants.

For example, the selection $\sigma_{\text{pname}=\text{Alice}}(\text{Transaction})$
returns the following relation:

\begin{table}[htbp]
    \centering
    \begin{tabular}{|c|c|c|c|}
        \hline
        tid     & pname & amount    & comment \\ \hline\hline
        1       & Alice & -1        & daily expense \\ \hline
        3       & Alice & 1         & friend\\ \hline
    \end{tabular}
    \caption{Selection of \textit{pname$=$Alice} on Transaction}
\end{table}

Similarly, the projection operator $\pi_{a1, \dots, a_n}(R)$ projects all columns
from a relation $R$ stated in the projection list $a1, \dots, a_n$, and
ignores all others. Consider the query $\pi_{\text{pname}}(\text{Transaction})$:

\begin{table}[htbp]
    \centering
    \begin{tabular}{|c|}
        \hline
        pname \\ \hline\hline
        Alice \\ \hline
        Bob   \\ \hline
        Carl  \\ \hline
        Dan   \\ \hline
    \end{tabular}
    \caption{Projection of \textit{pname} on Transaction}
\end{table}

Note that the projection removes fully duplicate rows, hence why Alice only exists once in
the resulting relation.

As the visualizations suggests, both the selection and projection return a relation.

\subsection{Grouping} \label{Grouping}

% We will use relational algebra to describe the actual grouping
% operations we want to perform.
% In chapter 4.2.1, \cite{Ram98} presents the selection
% operator, which selects a subset of rows from a relation.

% \begin{definition}
% Let $R$ be a relation. Then
% \[
%     \sigma_{c}({R}) := \set{e \in R | c(e)}
% \]
% is the selection on R with elements for which the condition $c$ holds true.
% \end{definition}

\newpage

\section{Simulation}

\subsection{Methods}
\subsection{Implementation}
\subsection{Results}

\section{Conclusion}
\subsection{Comparison}
\subsection{Future Work}