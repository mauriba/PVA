\documentclass[12pt,letterpaper]{article} % a. Set paper size to Letter, 8½ x 11. 

% Document settings and packages
%\usepackage[letterpaper,top=3.0cm, bottom=3.0cm, footnotesep=1.0cm]{geometry}
\usepackage[letterpaper,left=1.8in,right=1.8in,top=1in,bottom=1in]{geometry} % e. Set margins of 1 inch (2.54 cm.) on all four sides of the paper. 
\usepackage{mathptmx} % d. ...in a simple roman face except where indicated below (§3). 
\usepackage[frak=pxtx]{mathalpha}
\usepackage[singlespacing]{setspace} % Set line spacing to 1 throughout the document
\usepackage{fancyhdr}

\usepackage{amsthm}
\usepackage{braket}
\newtheorem{definition}{Definition}

\usepackage{hyperref}

\usepackage{listings}
\usepackage{xcolor}

\newcounter{codecounter}

\definecolor{codegreen}{rgb}{0,0.6,0}
\definecolor{codegray}{rgb}{0.5,0.5,0.5}
\definecolor{codepurple}{rgb}{0.58,0,0.82}
\definecolor{backcolour}{rgb}{0.95,0.93,0.95}

\lstdefinestyle{mystyle}{
    backgroundcolor=\color{backcolour},   
    commentstyle=\color{codegreen},
    keywordstyle=\color{magenta},
    numberstyle=\tiny\color{codegray},
    stringstyle=\color{codepurple},
    basicstyle=\ttfamily\footnotesize,
    breakatwhitespace=false,         
    breaklines=true,                 
    captionpos=b,                    
    keepspaces=true,                 
    numbers=left,                    
    numbersep=5pt,                  
    showspaces=false,                
    showstringspaces=false,
    showtabs=false,                  
    tabsize=2
}

\lstset{style=mystyle}

% Set the headheight to at least 14.49998pt
\setlength{\headheight}{14.49998pt}

% Optionally adjust \topmargin if necessary
% \addtolength{\topmargin}{-2.49998pt}
\usepackage{relsize}

\usepackage[bottom]{footmisc}
\usepackage{tabularx}
    
\pagestyle{empty}        % No page numbers

% Set paragraph indentation to 1.5cm
\setlength{\parindent}{2em}

%%%Using XeTeX (xelatex, lulatex):
%\usepackage{polyglossia}
%\usepackage{fontspec}
%\usepackage{xunicode}
%\usepackage{xltxtra}
\usepackage{url}
\usepackage[english,german]{babel}

\usepackage{graphicx}

%\setmainfont[Mapping=tex-text]{Linux Libertine O} %Falls nicht vorhanden müssen die LinLibertine-ttf-Dateien nach C:\windows\fonts verschoben werden

\usepackage{booktabs}    % For nice-looking tables
\usepackage{natbib}      % Citation support (required for crossrefs)
\usepackage{expex}
\bibpunct[:]{(}{)}{;}{a}{}{,} % Defaults for in-text citations
\usepackage{bibentry}    % Print individual references

\usepackage{acronym}
\usepackage{multicol}

% TODO: set authors name
%vorgelegt von\\
%Name des Verfassers/der Verfasserin\\
\author{Mauricé Ricardo Bärisch}

\usepackage{scrhack} % Recommended to avoid potential conflicts
\usepackage{microtype}

\usepackage[xindy,toc]{glossaries}
\makeglossaries
\usepackage{ragged2e}

% Set up hyphenation rules for the language package when mistakes happen
\babelhyphenation[english]{
an-oth-er
ex-am-ple
}

\begin{document}
\selectlanguage{english}

\begin{center}
    \includegraphics[width = 0.5\textwidth]{images/Berufliche_Hochschule_Hamburg_Logo_10.2020.svg}
\end{center}

\vspace*{10mm}
\begin{center}
\end{center}
\vspace*{10mm}

\title{Grouping 10,000 CSV records\\ \vspace{0,1cm} \large comparing PowerShell and MySQL}
\date{\vspace{-3ex}}
{\let\newpage\relax\maketitle}
\thispagestyle{empty}

\begin{center}
\begin{large}
\begin{Large}
Term Paper\\
\end{Large}
in course type Computer Science \\
\end{large}
\end{center}
\vspace{1cm}
\begin{center}
\begin{large}
Supervisor: Prof. Dr. Stefan Schiffner\\
\end{large}
\end{center}
\begin{center}
\begin{large}
% Advisor: Advisor's name\\
\end{large}
\end{center}

\begin{center}
\begin{large}
Submission Date: \date{\today} \\
\end{large}
\end{center}

\vspace{1,5cm}

\newpage
\tableofcontents
\newpage

\pagenumbering{arabic}

\section{Topic}

Performing aggregation and visualization on large datasets has become a major part of Information Technology.
Such datasets are often exported by a third party software or service provider, and arrive in a transmittable
format like JSON or CSV. Grouping such records can be challenging when exceeding the available
memory of the system.

We wrote a naive PowerShell script using the \verb+Group-Object+ command.
It aimed to group the records coming from a CSV export, but ended up
crashing the virtual environment it was running on, due to insufficient memory.

\section{Main contributions} \label{Main contributions}
Based on this experience, we can define the purpose of this work:
\begin{itemize}
    \item \textbf{Problem:} We need to aggregate CSV data bigger than the available memory
    \item \textbf{Objectives:}
    \begin{enumerate}
        \item Find a solution that can group large CSV data
        \item Proof that the solution works by running an
            experiment with data bigger than the available memory
    \end{enumerate}
    \item \textbf{Questions:}
    \begin{enumerate}
        \item Why does the \verb+Group-Object+ command not work on limited memory?
        \item What are requirements for a grouping algorithm in order to run on low memory?
    \end{enumerate}
\end{itemize}
As an alternative solution, we will consider MySQL, a Database management system.

\section{Background} \label{Background}

Databases, especially relational databases, are the gold standard for
storing and querying large data. After E. F. Codd presented a mathematical
model for relational databases in 1970 \cite{Codd70}, manufactors
adopted the principles and created SQL databases. Not only are
they capable of grouping large data, but there are also a lot of papers
and text books about them.

\subsection{External Sorting}
For example, \cite{Elma89} explains the external sorting strategies
applied to large datasets. External sorting algorithms run in
multiple so-called "runs" where parts of the data is being sorted
and written to the disk (sorting phase). After that,
the sorted parts are merged together in multiple iterations
(merge phase). This way, large data can be sorted by efficiently
making use of buffering and writing temporary results to storage.

\subsection{Coma-Separated Values (CSV)}
Specified in \cite{rfc4180}, CSV is a file format for storing a table in plain text. It has
the following characteristics:
1. each row starts on a new line
2. the values for each column are separated by a delimiter (usually a comma)
3. there may be a header line those values specify the names of the columns

\subsection{Docker}
Docker is an open platform for containerizing applications, thus making them independent
from the underlying infrastructure. We use Docker to build two comparable implementations
(PowerShell and MySQL) that group large CSV data.

\subsection{Container}
As explained on Docker's website \cite{docContainer}, a container isolates an application
from our operating system and files. It bundles the application source
code and required dependencies together, which makes the application very
portable.

\subsection{Aggregation}
Aggregation is the process of summarizing records using numerical functions, e.g.
calculate the total sum of a column. Aggregation also supports grouping the
records beforehand, allowing us to summarize groups of records.

But personally, I feel like \cite{Aren22} explains it better:
\begin{quote}
An extremely common feature of SQL queries
is the use of aggregation and grouping. Aggregation allows numerical functions to be
applied to entire columns, for example, to find the total salary
of all employees in a company. Grouping allows such columns to be split
according to a value of some attribute (...)
\end{quote}

\subsection{Relational Schema}
\cite{Aren22} defines the relation schema based on three characteristics:
\begin{itemize}
    \item \textbf{relation name}: the name of the table
    \item \textbf{relation attributes}: a list of uniquely identifiable attributes/columns
    \item \textbf{relation arity}: the number of attributes/columns
\end{itemize}
\cite{Schw10} defines the relation schema more formally as a database-wide mapping
from relation names to either their respective arity (unnamed perspective) or
their respective attributes (named perspective). Then, a row in a relation
is either defined as a tuple of values $a = (a_1,\dots, a_n)$ (unnamed perspective) or
a set of pairs $a = \set{(c_1, a_1),\dots, (c_n, a_n)}$, where each pair consists of
the column name $c_n$ and the value $a_n$ assigned to it (named perspective).

The named perspective is closer to DBMS practice, but requires columns to have a unique
name. We must ensure this in our exported CSV data to be able to make use of relational
databases.

\subsection{PowerShell}
PowerShell is a scripting language coming with Windows. It is mainly used for
automation and administration on Windows, but can also be used for general scripting tasks.

Modern PowerShell versions can be installed on other platforms as well. We will
use PowerShell within a ubuntu docker container for our experiment.

\section{Concept}

We use the following methods to find a solution that groups large CSV data:

\begin{enumerate} \label{Methods}
    \item We started off with a \textbf{systematic literature research}
        to explain the bottleneck of our PowerShell implementation
        and find an alternative solution.
    \item We build a \textbf{test}\footnote{For simplification,
        we just compare the aggregation result from both
        implementations on the same test data} to ensure that the new
        MySQL implementation does indeed aggregate the data in the same way
        our PowerShell implementation did.
    \item We run an \textbf{experiment} where both solutions run in
        Docker containers limited in memory. We expect the MySQL implementation
        to run slowly but without crashes due to its external sorting
        strategy.
\end{enumerate}

\section{Planned Structure}

\subsection{Lists and Abstract}

\paragraph{Abstract}
\textit{Performing aggregation and visualization on large datasets
has become a major part of Information Technology ...}

\paragraph{List of Tables}
We are going to show some tables representing a CSV file or SQL relation.

\paragraph{Listings}
For reproducible results, we provide all the code for the experiment in listings
or the appendix.

\paragraph{Glossary}
Because we use a lot of acronyms, e.g. "CSV", "DBMS" ...

\subsection{Introduction}

\paragraph{Motivation}
Explaining the circumstances in q.beyond why we had to group large CSV data.
\paragraph{Main contributions}
See \ref{Main contributions}

\subsection{Background}
Using \ref{Background} and future research to explain the two different kinds of grouping
(split into groups vs. aggregation on groups), pointing out that PowerShell's \verb+Group-Object+
command uses the first approach. Explaining how SQL databases like MySQL could help us solve
the problem (by using external sorting strategies). Making sure that we can use
relational databases for the data provided by the CSV export.

\subsection{Experiment}

\paragraph{Methods}
Explaining the methods used for this term paper (see \ref{Methods}).

\paragraph{Generating records}
Describing how we use an example CSV and bloat it up by repeatingly adding the same
rows to it, ending up with a ~400MB big CSV file we can use for our experiment.

\paragraph{Using Docker}
Explaining how and why docker is being used to create a reproducible experiment
and limiting the memory for it.

\paragraph{PowerShell Implementation}
Going into detail about the PowerShell implementation using \verb+Group-Object+
to group the CSV records.

\paragraph{MySQL Implementation}
Going into detail about the MySQL implementation. Also providing a script for initializing
the MySQL database with data corresponding to the contents of the CSV export.

\paragraph{Running the experiment}
Giving details about how to run the experiment.

\subsection{Discussion}

\paragraph{Evaluation}
Our experiment has a lot of quirks! We mention them here.

\paragraph{Findings}
Presenting our results after running the experiment a few times.

\paragraph{Future Work}
Given the motivation of this paper and the flaws of our experiment,
we hint to possible future work.

\subsection{Ending}

\paragraph{Appendix Scripts}
Providing scripts not discussed in the term paper
but required to run the experiment.

\paragraph{Bibliography}

\section{Schedule}

\begin{enumerate}
    \item Gather Topic Ideas (January 1 - January 7):
    Spending the first week brainstorming and gathering potential topics for the term paper.
    \item Decide on a Topic (January 8 - January 14):
    Narrowing down the list of ideas and selecting a final topic.
    Getting approval by my trainer.
    \item In-depth Research (January 15 - February 4):
    Conducting research, collecting relevant sources and information.
    \item Learn LaTeX (February 5 - February 18):
    Learning the basics of LaTeX.
    \item Learn Docker (February 19 - February 25):
    Spending time learning docker, focusing on how it can be used for the experiment.
    \item Outline the Paper (February 26 - March 3):
    Creating a detailed outline for the term paper.
    \item Implementing the Experiment (March 4 - April 7)
    Designing, implementing and running the experiment.
    \item Writing the Paper (April 8 - April 28)
    \item Proof-reading and Revisions (April 29 - May 12)
    \item Finalize and Format the Document (May 13 - May 17)
    \item Send in Finished Document (May 18)
\end{enumerate}

\clearpage

\addcontentsline{toc}{section}{Bibliography}
\pagestyle{fancy}

% \bibliographystyle{language-dt} %using language.bst
\bibliographystyle{IEEEtran}
\bibliography{bibliography} %bib-filename

\nocite{*} %List all bib-entries

\end{document}